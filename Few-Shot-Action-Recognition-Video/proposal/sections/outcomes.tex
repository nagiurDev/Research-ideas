\section{Expected Outcomes}


This research is expected to yield several significant outcomes and contributions to the field of few-shot action recognition:


\subsection{Improved Few-Shot Action Recognition Performance} 
We anticipate achieving a quantifiable improvement in few-shot action recognition accuracy. Specifically, we aim for a 10-15\% improvement in 5-way 1-shot and 5-way 5-shot accuracy on benchmark datasets like Something-Something-V2 and HMDB51, compared to state-of-the-art methods that use standard random data augmentations. This improvement will be measured using established evaluation protocols and will demonstrate the effectiveness of incorporating LLM-generated semantic information into the data augmentation process.

\subsection{A Novel Framework for LLM-Guided Data Augmentation}
This research will result in a novel framework for LLM-guided data augmentation. This framework will include the methodology for generating action descriptions using LLMs, mapping these descriptions to augmentation parameters, and applying the augmentations to training data. We expect this framework to be generalizable to other few-shot learning tasks in computer vision beyond action recognition, potentially impacting areas like object detection and image classification.

\subsection{Deeper Understanding of Semantic Augmentation} 
This work will contribute to a deeper understanding of the role of semantic information in data augmentation for few-shot learning. By analyzing the relationship between LLM-generated descriptions, augmentation parameters, and the resulting model performance, we will gain insights into how semantic information influences the effectiveness of data augmentation. This understanding could inform the development of more sophisticated and effective augmentation strategies in the future.

\subsection{Dissemination of Research Findings} We plan to disseminate our research findings through publications in top-tier computer vision conferences (e.g., CVPR, ICCV, ECCV) and journals (e.g., TPAMI, IJCV). This will make our contributions accessible to the broader research community and facilitate further advancements in the field.
