\section{Introduction}

Action recognition, the automatic identification of human actions from video data, is a cornerstone of computer vision with far-reaching applications in areas like surveillance, human-computer interaction, and robotics. Understanding and interpreting human behavior is crucial for creating intelligent systems capable of interacting seamlessly with the world. However, building robust action recognition models requires substantial amounts of labeled data, a resource often scarce when dealing with novel or specialized actions.\\

This data scarcity poses a significant challenge for few-shot action recognition, where models must learn to recognize new actions from only a handful of examples. Traditional deep learning models, typically trained on massive datasets, struggle to generalize effectively in these data-constrained scenarios. They are prone to overfitting the limited training data, resulting in poor performance on unseen examples.\\

Existing approaches to few-shot action recognition explore techniques like meta-learning and transfer learning. While these methods have shown promise, they often face limitations. Data augmentation, a common technique to artificially increase the training data size, often relies on random transformations like cropping, rotation, and jittering. These random augmentations may not generate semantically meaningful variations of the action, limiting their effectiveness in few-shot learning where understanding the nuances of the action is crucial.  While Large Language Models (LLMs) have shown remarkable abilities in understanding and generating human language, their potential for guiding data augmentation in computer vision remains largely unexplored.\\

This research proposes a novel approach to address these challenges by leveraging the power of LLMs to guide the data augmentation process for few-shot action recognition. We hypothesize that by using LLMs to generate descriptive and counterfactual variations of actions, we can create more targeted and effective synthetic training examples. Conditioning the data augmentation process on these LLM-generated descriptions will ensure that the generated variations are both diverse and semantically grounded, leading to improved model generalization.\\

Our research aims to develop a framework for LLM-guided data augmentation and evaluate its effectiveness in improving few-shot action recognition performance.  We will investigate different LLM architectures, prompting strategies, and augmentation techniques to optimize the generation of synthetic data.  Through rigorous experimentation on benchmark datasets, we aim to demonstrate the superiority of our proposed approach over existing few-shot learning methods.\\

This proposal details our research plan, outlining the methodology, expected outcomes, and required resources. We begin by reviewing the relevant literature on few-shot learning, data augmentation, and LLMs, followed by a detailed description of our proposed approach. We then present our evaluation plan and expected contributions, concluding with a timeline and a discussion of the necessary resources.