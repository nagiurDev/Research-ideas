\section{Methodology}
This section details the methodology for developing and evaluating the proposed LLM-guided data augmentation framework for few-shot action recognition. The methodology is divided into the following stages.


\subsection{LLM-Based Action Description Generation}
We will utilize a pre-trained LLM, such as GPT-3, fine-tuned on a dataset of action descriptions (e.g., a combination of existing video captioning datasets and manually curated descriptions). Prompt engineering will be crucial for generating diverse and relevant descriptions. We will explore different prompting strategies, including zero-shot prompting, few-shot prompting, and fine-tuning with specifically designed prompts to elicit variations, counterfactuals (e.g., "Imagine the person juggling while riding a unicycle"), and subtle nuances in action execution. The quality of generated descriptions will be assessed through human evaluation and by measuring their semantic similarity to ground truth descriptions using metrics like BLEU or METEOR.

\subsection{Augmentation Parameter Mapping}
We will develop a mapping model to translate LLM-generated descriptions into concrete augmentation parameters. This model will take the textual description as input and output a set of parameters for different augmentation techniques. We will explore different architectures for this mapping model, including recurrent neural networks (RNNs) and transformers, to effectively capture the semantic information in the descriptions. The model will be trained on a dataset of paired descriptions and augmentation parameters, potentially generated through a combination of manual annotation and programmatic generation based on keywords or phrases in the descriptions. The quality of the mapping will be evaluated by the realism and diversity of the resulting augmented samples, potentially using a perceptual similarity metric.

\subsection{Data Augmentation and Few-Shot Action Recognition Training}

The generated augmentation parameters will be used to apply transformations to the limited training examples in the few-shot setting. We will utilize a variety of augmentation techniques.

\begin{itemize}[wide, labelindent=20pt] 
    \item \textbf{Standard Transformations:}  Cropping, rotation, flipping, and jittering(color, temporal).
    \item \textbf{Advanced Techniques:} Mixup and CutMix, adapted for video data.
    \item \textbf{Generative Models:} We will explore the possibility of using GANs or diffusion models conditioned on the LLM descriptions to generate entirely new synthetic video segments.
\end{itemize}
For few-shot action recognition, we will employ a meta-learning approach, specifically MAML [1], due to its proven effectiveness in few-shot image classification and its adaptability to action recognition. The model architecture will be based on a convolutional neural network (CNN) designed for video processing, such as a 3D-CNN or a two-stream network. The model will be trained on the augmented dataset using standard optimization algorithms like Adam.

\subsection{ Evaluation}

We will evaluate the performance of our approach on benchmark datasets like Kinetics, Something-Something, and HMDB51 using a standard few-shot learning evaluation protocol. Performance will be measured using standard metrics, including:
\begin{itemize}[wide, labelindent=20pt] 
    \item \textbf{N-way K-shot Accuracy:} The primary metric for few-shot learning.
    \item \textbf{Precision, Recall, F1-score:} To provide a more comprehensive evaluation.
\end{itemize}
We will compare our approach against relevant baseline methods, including:
\begin{itemize}[wide, labelindent=20pt] 
    \item \textbf{Few-shot learning methods with standard random augmentations:} To demonstrate the benefit of LLM guidance.
    \item \textbf{State-of-the-art few-shot action recognition methods:} To assess the competitiveness of our approach.
\end{itemize}

\subsection{Implementation Details}

We will use Python with deep learning libraries like PyTorch or TensorFlow. Experiments will be conducted on a high-performance computing cluster with GPUs. The code will be made publicly available to ensure reproducibility.

\subsection{ Potential Challenges and Limitations}
\begin{itemize}[wide, labelindent=20pt] 
    \item \textbf{Generating high-quality and diverse descriptions:} The success of our approach relies heavily on the quality of the LLM-generated descriptions. We will address this by carefully designing prompting strategies and exploring different LLM architectures.

    \item \textbf{Computational cost:} Training LLMs and generative models can be computationally expensive. We will explore efficient training strategies and utilize pre-trained models whenever possible.
    \item \textbf{Overfitting to the LLM's biases:} The LLM might introduce biases into the augmented data. We will address this by carefully evaluating the generated descriptions and augmentations and by exploring techniques to mitigate bias in LLMs.
\end{itemize}