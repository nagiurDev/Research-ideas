\section{Resources Required}


\subsection{Hardware }
 We will require access to a high-performance computing cluster with at least 4x NVIDIA RTX A6000 GPUs or equivalent, providing sufficient computational power for training deep learning models, particularly the computationally intensive fine-tuning of large language models and processing of video data. A minimum of 128GB of RAM is required for efficient processing of large datasets and models. We will need approximately 5TB of storage for storing datasets, trained models, and intermediate results.

\subsection{ Data Resources}

The following publicly available datasets will be used:
\begin{itemize}[wide, labelindent=20pt]
    \item \textbf{Something-Something V2:} For evaluating performance on human-object interaction and fine-grained actions.
    \item \textbf{HMDB51:} For evaluating performance on a smaller, well-established action recognition dataset.
    \item \textbf{Kinetics:} For potential use in pre-training or supplementary experiments.
\end{itemize}

Data will be stored on the university's high-performance computing cluster's storage system, ensuring secure and efficient access.

\subsection{ Software and Tools}

\begin{itemize}[wide, labelindent=20pt]
    \item \textbf{LLM Access:} We will primarily utilize publicly available pre-trained LLMs (e.g., GPT-Neo, CLIP) and explore fine-tuning them for action description generation. We will consider using paid API access to more powerful LLMs (e.g., GPT-3) if necessary, and budget for this will be requested if required.
    \item \textbf{Data Augmentation Tools:} Existing data augmentation libraries within TensorFlow/PyTorch and specialized libraries like Albumentations will be used for implementing and applying various augmentation techniques.
    \item \textbf{Evaluation Tools:} Standard evaluation metrics will be implemented using Python and relevant libraries like Scikit-learn.
    \item \textbf{Version Control:} Git will be used for version control and collaborative code development, hosted on a platform like GitHub or GitLab.

\end{itemize}


